% !TEX root = ./../main.tex
\chapter{Model of biological membrane and monolayer--protected NP}

\section{NanoParticle model}
Monolayer--protected metal \acp{NP} have the advantage of having a well defined molecular structure. That is, mono--dispersed solutions can be synthesized and structurally determined. In particular, \acp{NP} with a gold core can be stabilized by thiols\footnote{The thiol (R--SH) group is like the alcohol (R--OH) group but it is less polar respect to the second due the lower electronegativity of the sulfur respect to oxygen.}, which stably bind to the Au surface via Au-S bonds. Thiolated Au\acp{NP} are air stable, electrochemically stable and thermally stable compounds []. Several stable thiolated Au\acp{NP} are identified, differing in size of the gold core and number of ligands. In particular we consider a {Au$_{144}$(SR)$_{60}$} \ac{NP}, where R is the aliphatic chains of the thiol compounds. 

Changing the composition of the aliphatic chains bonded to the thiol group different properties of the thiolated Au\ac{NP} can be achieved, such as different net charge, different level of hydrophobicity, different size and so on. In particular, as we shall see later, in the model we will use we consider only \ac{OT} and \ac{MUS} ligands that cover the Au\ac{NP} core with a monolayer in different composition and different surface arrangement. To overcome the computational cost of an atomistic model we use a \martini \ac{CG} model of the \ac{OT} and \ac{MUS} ligands. 

The model of \ac{NP} gold core and the \martini model of the \ac{OT} and \ac{MUS} ligands are developed by Federica Simonelli \etal\, in \cite{ourPaper} and \cite{simonelliThesis}, and the reader is addressed to they for a more detail discussion about the model and the parameterization.

% gold core --> Thyiol passivated --> ligands
%Ligand Composition: Patched (1:1), Random (1:1) (1:2)
%Different level of hydrophobicity
%Different ligand charge: anionic/cationic NP

\subsection{Passivated gold core}
The gold core is composed of $144$ atoms which displays icosahedral symmetry and it is made of three bulk shell with $12$, $42$ and $60$ atoms, respectively. A surface shell of $30$ atoms completes the gold cluster structure. Then $60$ sulphur atoms, which blind the aliphatic chains (R) to the gold core, are bounded to the gold atoms on the surface through the typical bond structure RS--Au--SR. The shell construction is shown in figure~(\ref{fig:goldShell}).
\begin{figure}[!ht]
	\centering
	\includegraphics[width=0.8\textwidth]{./img/goldShell}
	\caption{First three frame: the concentric $12$--(yellow), $42$--(red) and $60$--(blue) atom gold internal shell, surrounded (last frame) by $30$ gold (red small) and $60$ sulphur (yellow small) surface atoms. The R chains is not show. Taken from \cite{corePassivated}.}
	\label{fig:goldShell}
\end{figure}

The resulting diameter of the gold core is about $2$~nm. When passivated by thiols, its overall size depends on the length of the aliphatic chains bound to the sulphur atoms. The monolayer--protected Au\ac{NP} we will consider have a total diameter of about $4$~nm.

Despite the computational cost associated to atomistically describe the gold cluster, all gold atoms are taken into account in order to allow us future studies on heating effect on lipid bilayer. Thus the bonds between gold atoms are allowed to vibrate. As we have seen in a previous section, a many--body potential should be used. Instead, as we are interested in the vibrational modes of the core atoms, 
%Federica Simonelli in her thesis work, found that
a more efficient way is to use an elastic network associated to the potential energy
\begin{equation*}
	U = \frac{1}{2}\sum_i \sum_{j\ne i}k_{ij}(r_{ij} - r_{ij}^0)^2
\end{equation*}
where $r_{ij}$ is the distance and $k_{ij}$ is the bond constant for $i-j$ atom pair. The bond constants is assigned so as to reproduce the vibrational spectrum of the gold core as provided by the many--body Gupta potential. The constant is assigned to $k = 32500$~kJ/(mol\,nm$^2$) for surface atoms and $k = 11000$~kJ/(mol\,nm$^2$) for bulk atoms. While the equilibrium distances were derived from []. A gold atom is considered as bulk atom if it has at least nine gold atom neighbors otherwise as a surface atom. A gold atom is considered neighbor to another if it lie within a shell of radius $0.35$~nm centered to the considered atom. In figure~(\ref{fig:goldNetwork}) the gold elastic network is shown.
\begin{SCfigure}
	\centering
	\includegraphics[width=0.3\textwidth]{./img/goldNetwork}
	\caption{Elastic network, represented by the stick bond, among gold atoms (pink). In cyan a surface atoms and its neighbors. In blue a group of bulk atoms and its neighbors. Taken from \cite{simonelliThesis}.}
	\label{fig:goldNetwork}
\end{SCfigure}

An elastic network with bond constant of $1250$~kJ/(mol\,nm$^2$) is also used to model sulphur atoms on the surface of the \ac{NP} cluster with a cutoff distance of $0.55$~nm to select the sulphur neighbor atoms. The interaction between a sulphur atom and its nearest gold atoms is modeled through a harmonic potential with a bond constant of $32500$~kJ/(mol\,nm$^2$) and equilibrium distances given by []. In order to prevent the penetration of the sulphur atoms into gold core, a repulsive potential of the form $C/r^{-12}$ where $C = 0.92953\cdot 10^{-6}$~(kJ\,nm$^{12}$)/mol models the non--bonded interaction between gold and sulphur atoms which are not involved in any of the previous bonds. In figure~() sulphur passivated Au\ac{NP} cluster with the complete elastic network is shown.

For a more comprehensive discussion about the model of the \ac{NP} gold core the reader is addressed to thesis work of Federica Simonelli \cite{simonelliThesis}.

% some information about gold core: dimensions, number of atoms, model used, elastic network, shel construction

\subsection{Functionalizing ligands}
Our Au\ac{NP} core is functionalized with \ac{MUS} and \ac{OT} ligands. The former is a charged compounds made of eleven \ac{CH2} groups and a charged terminal \ac{SO4-} group. The charged terminal group make it partially hydrophilic. The latter, instead, is completely hydrophobic and it is made by seven \ac{CH2} groups and one \ac{CH3} terminal group. Using both hydrophilic and hydrophobic groups guarantees that such \acp{NP} are stable, that is, does not aggregate, in both aqueous and non--aqueous environments.

%\subsection{OT ligands}
\paragraph{\textbf{OT Model}} Two \martini beds of type C$_1$ model the eight carbon atoms of the \ac{OT} backbone and their hydrogen atoms. The chemical structure and the resulting \ac{CG} \martini model is shown in figure~(\ref{fig:ot}). The first bead of each \ac{OT} ligands is bound to a sulphur atom via a harmonic potential with a bond constant of $1250$~kJ/(mol\,nm$^2$) and equilibrium length of $0.47$~nm. The second bead is connected to the first by same bond potential. An angle potential as in equation~\eqref{eq:martiniAngle} is used among the three particles. Parameters are fixed in accordance with the standard \martini ones (see section~\ref{sec:martiniPotential}). Moreover a purely repulsion potential, as described in the previous section, is used between C$_1$ beads and gold and sulfur atoms to prevent the co--penetration.
\begin{SCfigure}
	\includegraphics[width=0.35\textwidth]{./img/OT/OT}
	\caption{Top: chemical structure of the \acs{OT} ligands. Bottom: \acs{CG} \martini model (red: C$_1$ bead and yellow: sulfur atom).}
	\label{fig:ot}
\end{SCfigure}

%\subsubsection{MUS ligands} %Anionic an the cationic???
\paragraph{\textbf{MUS Model}} Three \martini beads of type C$_1$ model the hydrophobic chain even if one of them groups three carbon atoms instead of four. The charged group is modeled as a Q$_\text{da}$ beads with a charge of $-\mathsf{e}$. The chemical structure and the resulting \ac{CG} \martini model is shown in figure~(\ref{fig:mus}). Even in this case the first bead of a \ac{MUS} ligand is bounded to the sulphur atom through a harmonic potential with the same parameter: bond constant of $1250$~kJ/(mol\,nm$^2$) and equilibrium length of $0.47$~nm. The same potential is used to bind all other beads to the previous one. An angle potential as in equation~\eqref{eq:martiniAngle} is used among the sulfur atom, the first C$_1$ and second C$_1$, among the first, the second and the third C$_1$ beads and so on for all four beads. Parameters are fixed in accordance with the standard \martini ones (see section~\ref{sec:martiniPotential}). As in the \ac{OT} ligand model, a purely repulsion potential, as described in the previous section, is used between C$_1$ beads and gold and sulfur atoms to prevent the co--penetration. The same applies between Q$_\text{da}$ beads and gold and sulfur atoms.
\begin{SCfigure}
	\includegraphics[width=0.45\textwidth]{./img/MUS/MUS}
	\caption{Top: chemical structure of the \acs{MUS} ligands. Bottom: \acs{CG} \martini model (green: Q$_\text{da}$ negatively charged bead, red: C$_1$ bead and yellow: sulfur atom).}
	\label{fig:mus}
\end{SCfigure}

\paragraph{\textbf{level of hydrophobicity}} The Au\ac{NP} core can be functionalized with both ligands in different composition. In particular varying the ratio between the \ac{OT} and \ac{MUS} ligands different levels of hydrophobicity can be reached: it decrease augmenting the number of charged ligands. Two surface composition is considered: (\ac{MUS}:\ac{OT} $1$:$1$) and (2:1), but the former is the main used in this thesis work.

\paragraph{\textbf{surface arrangements}} The arrangements of the ligands on the Au\ac{NP} surface can be made in two possible way: random or predetermined scheme. In particular for the second possibility we use a striped scheme: the \ac{NP} surface is divided in three striped, the external two stripes are cover with \ac{MUS} ligands while the central one with \ac{OT} ligands. For this thesis work we consider three type of \ac{NP}: striped (\ac{MUS}:\ac{OT} $1$:$1$), random (\ac{MUS}:\ac{OT} $1$:$1$) and random (\ac{MUS}:\ac{OT} $2$:$1$). In figure~(\ref{fig:coating}) the different coatings for the \ac{NP} core is shown.

\begin{figure}[!ht]
	\centering
	\includegraphics[width=0.9\textwidth]{./img/coatings/coat}
	\caption{Au\acs{NP} with different ligands surface arrangements and composition. From left to right: striped (\ac{MUS}:\ac{OT} $1$:$1$), random (\ac{MUS}:\ac{OT} $1$:$1$) and random (\ac{MUS}:\ac{OT} $2$:$1$). Hydrophobic beads are shown in red while the negatively charged beads are green.}
	\label{fig:coating}
\end{figure}


\section{Model biological membrane}

\subsection{Real membrane}

\subsection{MARTINI model}

\section{NP--Membrane interaction}

\subsection{Three--stage process}

\subsection{Preliminary free energy calculations}