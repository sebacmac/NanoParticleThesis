% !TEX root = ./../main.tex
\chapter{Summary and conclusions}
\label{chap:conclusions}
% Brevemento cosa ho fatto

% Risultati più importanti:
% abbiamo definito un protocollo per il calcolo delle buche di andata e ritorno separatamente, il fatto che i percorsi siano diversi. Prospettive di sviluppo: vedere come cambia tirando giù altri ligandi, CV diversi nella speranza di trovare meccanismi alternativi e vantaggiosi, capire  se è favorito o meno dalla presenta di altre NP nella membrana
%
% Convinti che martini STD rimane qualitativamente indispensabile ma quantitativamente insufficente per via dell'int elettrostatoca: usare PME PW. Efficienza tra i due modelli.
%
% La differenza tra Random e striped rimane confermato: studiare ligandi diversi, NP con dimensioni più grandi che permattano di giocare con i diversi arrangement.

This thesis is focused on study the role played by the electrostatic interaction and the different levels of accuracy of its treatment in the one anchor process --- from the \ac{NP} in the hydrophobic state to the state in which one charged \ac{MUS} ligand is anchored to the opposite leaflet --- an important stage in the anionic, passivated \ac{AuNP} interaction with biological membrane. We used computational methods, \ac{MD} with a popular \ac{CG} \ac{FF} for biomolecular applications, the \martini \ac{FF} \cite{Martini}, to simulate the system and advanced sampling techniques, metadynamics \cite{MetadParrinello}, to perform free energy calculations: speed--up the anchoring process and try to estimate the free energy profile involved in the forward process, the anchoring, and the backward process, the dis--anchoring of the charged ligand terminal. The achieved results, with a more sophisticated \ac{CG} model, allow us also to investigate the membrane deformation and the molecular processes involved in the charged ligand translocation, such as the leaflets deformation due to the dragging effect of water and lipid heads by the charged ligand terminal. In the following, we summarize the main achieved results and the future perspectives.

\begin{enumerate}[label=\roman*.]
	\item \textit{Calculation of the forward and backward energy barriers}, separately. With the comparison we made between different \ac{CG} models: the \ac{STD} \martini, \martini with \ac{PME} and \martini with both \ac{PME}$+$\ac{PW} and the atomistic one, we understand, at the contrary of what we thought, that \textit{the path way for the forward and backward processes are evidently different causing a sampling error in the whole \ac{FES} of the translocation process}. In particular, for the striped (\acs{MUS}:\acs{OT} $1$:$1$) \ac{NP} both the barriers are almost identically and equal to about $100$~kJ/mol; for the random (\acs{MUS}:\acs{OT} $1$:$1$) \ac{NP} the forward barrier is equal to about $60$~kJ/mol while the backward one is higher than the forward of about $35$~kJ/mol. The future perspectives are related in more studies of such free energy profiles: trying to pull down other ligands if one is already present in the anchored state and understand if the translocation process is helped or hindered with other \acp{NP} on the lipid bilayer. We understand that the used \ac{CV} suffers of some issues that leads in systematic sampling errors due to the hidden energy barriers. Hence can be of interest trying to find another \ac{CV} or try to increase the dimensionality of it using more than one \acp{CV}: as example, we could use the number of contacts between the charged ligand terminal and the lipid or both the \acp{CV}. These attempts can also be useful in finding other molecular mechanism that make the translocation process favorable.%
	
	\item Thanks to its extremely high computational efficiency, we believe that \textit{the \ac{STD}} \martini \textit{\ac{FF} remains indispensable for qualitatively results} and for understand the first molecular basis of a process, without all the computational issues related to a finer model. Unfortunately, \textit{it is insufficient for obtain quantitative results}, especially of such processes that strongly depends on the electrostatic interaction. Despite the slight worsening of the computational performance, we have to include the \ac{PME} method for a long--range treatment of the electrostatic interaction and the \ac{PW} model in order to cancel out the use of an implicit medium that change the behavior of an hydrophobic environment, such like the core of a lipid membrane, and to achieve a better behavior of the water solvent.%
			
	\item \textit{We have confirmed the trend between the striped and the random \acp{NP}}, as observed with the \ac{STD} \martini \ac{FF} in \cite{ourPaper}. The striped \ac{NP} presents a forward energy barrier higher than the random one of about $40$~kJ/mol. The future perspectives are related to investigate how the \ac{FES} change with a multicomponent lipid membrane, for example adding cholesterol, that increases the rigidity of the membrane; how the free energy profile change with different degrees of hydrophobicity of the \ac{NP}, using ligands with different length; different kind of ligands; different surface arrangements; or using larger \acp{NP} that allow to play with this different configurations of the ligands.%
\end{enumerate}