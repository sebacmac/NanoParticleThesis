% !TEX root = ./../main.tex
\chapter{Model of cell membrane and monolayer--protected NP}
\label{chap:tre}
In the first part of this chapter we will present and describe the model of the charged monolayer--protected 
\ac{AuNP} developed by Federica Simonelli and co--workers in \cite{simonelliThesis} and \cite{ourPaper}. The gold 
core is treated by an all atoms representation, while the ligands are modeled at \ac{CG} level by the \martini 
\ac{FF}. In the second part of the chapter we will describe the most important physical and chemical features of 
the cell membrane. Then we will summarize the characteristics of the \ac{CG} model used to treat it. Finally, we 
will describe the interaction mechanism and some preliminary thermodynamic results about the interaction of the 
\ac{AuNP} and the model cell membrane, as outlined in \cite{ourPaper}. For a more precise discussion about the 
\ac{NP}--membrane interaction and the models parameterization the reader is addressed to the work of Federica 
Simonelli \etal\, \cite{ourPaper} and her thesis work \cite{simonelliThesis}. For more details about the gold core 
used, its properties, equilibrium structure and so forth, the reader is addressed to the work of Lopez-Acevedo 
\etal\, \cite{clusterEquilibrium} while for a general discussion about thiolated \acp{AuNP} to the work of 
Häkkinen \cite{corePassivated}.

\section{NanoParticle model}
% gold core --> Thyiol passivated --> ligands
%Ligand Composition: Patched (1:1), Random (1:1) (1:2)
%Different level of hydrophobicity
%Different ligand charge: anionic/cationic NP
Monolayer--protected \acp{AuNP} have the advantage of having a well defined molecular structure. That is, 
mono--dispersed solutions can be synthesized and structurally determined. In particular \acp{AuNP} are 
biocompatible, have a facile surface chemistry and efficiently convert light into heat. Monolayer--protected 
\acp{AuNP} have a definite mass and molecular composition, and their structure is stabilized by the covalently 
bound ligands shell. Most commonly they are thiolated compounds because they covalently bound to the gold \ac{NP} 
by Au--S surface bonds, that is, a robust but modifiable interaction, crucial in stabilizing the \ac{NP} and 
transmitting electronic interactions between gold and sulfur--containing organic molecules \cite{corePassivated}. 
Moreover monodisperse \acp{AuNP} can be synthesized in the $1-4$~nm range. Subtle changes of size, structure and 
ligand compositions and arrangements, can affect \ac{NP} properties such as their optical properties, important 
for biological sensing and therapeutics. 
%Thiolated \acp{AuNP} are air stable, electrochemically stable and thermally stable compounds [].
Several stable thiolated \acp{AuNP} are identified, differing in size of the gold core and number of ligands 
\cite{corePassivated}. In this thesis work we will consider the {Au$_{144}$(SR)$_{60}$} thiolated \ac{AuNP}, where 
R are the aliphatic chain of the thiol compounds. The equilibrium structure of the gold core is described by 
\textit{ab--initio} calculation in \cite{clusterEquilibrium}. 

Changing the composition of the aliphatic chains bonded to the thiol group, different properties of the thiolated 
\ac{AuNP} can be achieved, such as different net charge, different level of hydrophobicity, different size and so 
on. In particular, as we shall see, in the model we will use we consider only \ac{OT} and \ac{MUS} ligands that 
cover the \ac{NP} gold core with a monolayer with different compositions and surface arrangements. 
%To overcome the computational cost of an atomistic model we use a \martini \ac{CG} model of the \ac{OT} and \ac{MUS} ligands. 

%The model of the \ac{AuNP} and the \martini model of the \ac{OT} and \ac{MUS} ligands are developed by Federica Simonelli \etal\, in \cite{ourPaper}, and the reader is addressed to it for a more detail discussion about the model and the parameterization.

\subsection{Passivated gold core}
% some information about gold core: dimensions, number of atoms, model used, elastic network, shel construction
The gold core is composed of $144$ atoms, it has icosahedral symmetry and it is made of three bulk shell with 
$12$, $42$ and $60$ atoms, respectively. A surface shell of $30$ atoms completes the gold cluster structure. Then 
$60$ sulphur atoms, which bind the aliphatic chains (R) to the gold core, are bounded to the gold atoms on the 
surface through the typical bond structure RS--Au--SR. The shell construction is shown in 
figure~(\ref{fig:goldShell}).
\begin{figure}[!ht]
	\centering
	\includegraphics[width=0.8\textwidth]{./img/goldShell}
	\caption{First three frame: the concentric $12$--(yellow), $42$--(red) and $60$--(blue) atom gold internal shell, surrounded (last frame) by $30$ gold (red small) and $60$ sulphur (yellow small) surface atoms. The R chains are not shown. Taken from \cite{corePassivated}.}
	\label{fig:goldShell}
\end{figure}

The resulting diameter of the gold core is about $2$~nm. When passivated by thiols, its overall size depends on 
the length of the aliphatic chains bound to the sulphur atoms. The monolayer--protected \acp{AuNP} we will 
consider have a total diameter of about $4$~nm.

Despite the computational cost associated to atomistically describe the \ac{NP} core, all gold and sulfur atoms 
are taken into account. Bonds between gold atoms and sulfur atoms are allowed to vibrate. As we have seen in a 
previous section, a many--body potential should be used. Instead, as we are interested in the vibrational modes of 
the core atoms, 
%Federica Simonelli in her thesis work, found that
a more efficient way, as shown in figure~(\ref{fig:coreNetwork}), is to use an elastic network associate the 
potential energy
\begin{equation*}
	U = \frac{1}{2}\sum_i \sideset{}{'}\sum_{j\ne i}k_{ij}(r_{ij} - r_{ij}^0)^2
\end{equation*}
where $r_{ij}$ is the distance, $k_{ij}$ is the bond constant for $i-j$ atom pair and the prime indicates that 
only neighbor atoms within a certain cutoff are considered. 
\begin{figure}[h!t]
	\centering%
	\subfloat[Gold elastic network]{%
		\includegraphics[width=0.35\textwidth]{./img/goldNetwork}%
	}\qquad%
	\subfloat[\acs{AuNP} cluter]{%
		\includegraphics[width=0.32\textwidth]{./img/NPCluster}%
	}
	\caption{Left: gold elastic network. In cyan a surface atom and its neighbors; in blue bulk atom and its neighbors. Right: \acs{AuNP} cluster. The elastic network for both gold and sulfur atoms are represented by sticks. Taken from \cite{simonelliThesis} and \cite{ourPaper}.}
	\label{fig:coreNetwork}
\end{figure}

%The bond constants is assigned so as to reproduce the vibrational spectrum of the gold core as provided by the many--body Gupta potential. 
The bond constant is assigned to $k = 32500$~kJ/(mol\,nm$^2$) for Au--Au surface atoms, $k = 11000$ kJ/(mol\ 
nm$^2$) for Au--Au bulk atoms, $k = 1250$~kJ/(mol\,nm$^2$) for S--S atoms and $k = 32500$~kJ/(mol\,nm$^2$) for 
Au--S bonds, as summarized in table~(\ref{tab:NPConstants}). Instead, the equilibrium distances are derived from 
\textit{ab--initio} data in \cite{clusterEquilibrium}. Moreover, to prevent the penetration of other particles 
inside the \ac{NP} core, a purely repulsive interaction of the form $C/r^{-12}$ where 
$C = 0.92953\cdot 10^{-6}$~(kJ\,nm$^{12}$)/mol, is added between gold and sulfur atoms, gold and all other 
particles and sulfur and all other particles. A gold atom is considered as bulk atom if it has at least nine gold 
atom neighbors otherwise as a surface atom. Two gold atoms $i$ and $j$ are neighbors if their distance is 
$r_{ij} \le 0.35$~nm. Instead, two sulfur atoms are considered neighbors if they lie in a sphere shell of radius 
$0.55$~nm, i.e. each sulfur atom have at least five neighbors.
\begin{table}[h!t]
	\centering
	\begin{tabular}{lr}
		\toprule
		Bond			 & $k$\,\footnotesize[kJ/(mol\,nm$^2$)] \\ \toprule
		Au--Au (bulk) 	 & $11000$ 	\\ \midrule
		Au--Au (surface) & $32500$ 	\\ \midrule
		Au--S			 & $32500$ 	\\ \midrule
		S--S			 & $1250$ 	\\ \bottomrule
	\end{tabular}
	\caption{Summary of the bond constants for the elastic network of the \acs{NP} core.}
	\label{tab:NPConstants}
\end{table}

\subsection{Functionalizing ligands}
Our \ac{AuNP} core is functionalized with \ac{MUS} and \ac{OT} ligands as shown in figure~\ref{fig:figands}. 
\ac{MUS} is a charged compound made of an alkyl chain (\acs{CH2})$_{11}$ and a charged terminal \acs{SO4-} group. 
The charged terminal group makes \ac{MUS} partially hydrophilic. \ac{OT}, instead, is completely hydrophobic and 
it is made by an alkyl chain (\acs{CH2})$_{7}$ and one \acs{CH3} terminal group. Using both hydrophilic and 
hydrophobic groups guarantees that \acp{NP} are stable, that is, they do not aggregate in aqueous environments.
\begin{figure}[!ht]
	\centering
	\subfloat[\acs{OT} ligand]{%
		\includegraphics[width=0.3\textwidth]{./img/OT/OT}%
		\label{fig:ot}%
	}%
	\qquad\qquad%
	\subfloat[\acs{MUS} ligand]{%
		\includegraphics[width=0.35\textwidth]{./img/MUS/MUS}%
		\label{fig:mus}%
	}%
	\caption{Top: chemical structure. Bottom: \acs{CG} \martini model (red: C$_1$ bead, green: Qda negatively charged bead and yellow: sulfur atom).}
	\label{fig:figands}
\end{figure}

\paragraph{\textbf{OT Model}} Two \martini beads of type C$_1$ model the eight carbon atoms of the \ac{OT} 
backbone and their hydrogen atoms. The chemical structure and the resulting \ac{CG} \martini model is shown in 
figure~(\ref{fig:ot}). The first bead of each \ac{OT} ligand is bound to a sulphur atom via a harmonic potential 
with a bond constant of $1250$~kJ/(mol\,nm$^2$) and equilibrium length of $0.47$~nm. The second bead is connected 
to the first by the same bond potential. An angle potential as in equation~\eqref{eq:martiniAngle} is used among 
the three particles. Parameters are fixed in accordance with the standard \martini parameters for alkanes. 
%Moreover a purely repulsive potential, as described previously, is used between C$_1$ beads and gold and sulfur 
%atoms to prevent the penetration of the \ac{NP} core.

\paragraph{\textbf{MUS Model}} Three \martini beads of type C$_1$ model the hydrophobic chain of the \ac{MUS} 
ligand. The charged group is modeled as a Qda bead with a charge of $-\mathsf{e}$. The chemical structure and the 
resulting \ac{CG} \martini model is shown in figure~(\ref{fig:mus}). Even in this case the first bead of a 
\ac{MUS} ligand is bound to the sulphur atom through a harmonic potential with the same parameter: bond constant 
of $1250$~kJ/(mol\,nm$^2$) and equilibrium length of $0.47$~nm. The same potential is used to bind all other beads 
to the previous one. An angle potential as in equation~\eqref{eq:martiniAngle} is used among the sulfur atom, the 
first C$_1$ and second C$_1$, among the first, the second and the third C$_1$ beads and so on for all four beads. 
Parameters are fixed in accordance with the standard \martini parameters for alkanes.%As in the \ac{OT} ligand 
%model, a purely repulsive potential, as described previously, is used between C$_1$ beads and gold and sulfur 
%atoms to prevent the \ac{NP} penetration. The same applies between Qda bead and gold and sulfur atoms.

\paragraph{\textbf{level of hydrophobicity}} The \ac{AuNP} core can be functionalized with both ligands at 
different composition. In particular varying the ratio between the \ac{OT} and \ac{MUS} ligands different levels 
of hydrophobicity can be reached. Two surface compositions will be considered in this thesis work: 
(\ac{MUS}:\ac{OT} $1$:$1$) and (2:1), the former is the main used in this thesis work. This choice is ?? to the 
possibility to compare to previous experimental and simulation data \cite{experimentMaccarini}.

\paragraph{\textbf{surface arrangements}} The ligands on the \ac{AuNP} surface can be arranged in two possible 
ways: randomly or with a predetermined scheme. We will consider both \acp{NP} with a random ligand arrangement and 
\acp{NP} with a striped ligand arrangement. The striped scheme is obtained dividing the \ac{NP} surface in three 
stripes: the external two stripes are covered with \ac{MUS} ligands while the central with \ac{OT} ligands. For 
this thesis work we consider three type of \acp{NP}: striped (\ac{MUS}:\ac{OT} $1$:$1$), random (\ac{MUS}:\ac{OT} 
$1$:$1$) and random (\ac{MUS}:\ac{OT} $2$:$1$). In figure~(\ref{fig:coating}) the different coatings for the 
\ac{NP} core is shown.

\begin{figure}[!ht]
	\centering
	\subfloat[striped ($1$:$1$)]{
		\includegraphics[width=3.3cm]{./img/coatings/striped}
	}%
	\qquad
	\subfloat[random ($1$:$1$)]{
		\includegraphics[width=2.9cm]{./img/coatings/random11}
	}%
	\qquad
	\subfloat[random ($2$:$1$)]{
		\includegraphics[width=2.7cm]{./img/coatings/random21}
	}%
	\caption{Au\acs{NP} with different ligands surface arrangements and composition. From left to right: striped (\ac{MUS}:\ac{OT} $1$:$1$), random (\ac{MUS}:\ac{OT} $1$:$1$) and random (\ac{MUS}:\ac{OT} $2$:$1$). Hydrophobic beads are shown in red while the negatively charged beads are green.}
	\label{fig:coating}
\end{figure}

\section{Cell membranes}

\subsection{Real cell membranes}
The cell membrane or cytoplasmic membrane is a biological membrane that separate the interior and the external environments of a living cells. The basic function is to protect cells from its surroundings but it also takes the function of a ``customs'' in order to allow the cells to exchange chemical compounds from and to the external environment. To fulfill at this role the cell membrane is a crowded environment consisting of phospholipids, glycolipids, carbohydrates proteins and so on as we can see from figure~(\ref{fig:cellMembrane}).
\begin{figure}[!ht]
	\centering
	\includegraphics[width=0.9\textwidth]{./img/cellMembrane}
	\caption{Schematic representation of a biological cell membrane.}
	\label{fig:cellMembrane}
\end{figure}

The skeleton of a cell membrane is a bilayer sheet made of \textit{phospholipids}. They are a kind of lipids made of a polar or charged head, which is hydrophilic and one or two fatty acid hydrocarbon chains, often called lipid tails, which are instead hydrophobic; phospholipids are then amphiphilic molecules. This amphiphilic nature play a key role in the bilayer formation. In fact the formation of a bilayer in water is a self--assembly process driven by the hydrophobic effect which acts so as to minimize the number of hydrophobic contact between water and lipid tails. Moreover the bilayer sheet is not the only possible structure, based on the concentration of the phospholipids in water and on the temperature the self--assembly process can lead to three main different structures: bilayer sheet, liposome and micelle. In figure~(\ref{fig:lipidsStructures}) a schematic representation of this three different structures is shown.
\begin{SCfigure}[][!ht]
	\includegraphics[width=0.3\textwidth]{./img/lipidsStructures}
	\caption{Cross--section view of the structures that can be formed by a self--assembly process of phospholipids in aqueous solution.}
	\label{fig:lipidsStructures}
\end{SCfigure}

A real lipid bilayer often contains hundred of different lipid species. They differ in the length of the hydrocarbon chains, in the degree of unsaturation, i.e. in the number of double bond in the hydrocarbon chains, and in different composition of the head that can be polar or non--polar. There are two main classes of phospholipids that make a cell membrane of animals: glycerophospholipid (phosphatidyl--choline, phophatidyl--ethanolamine, phophatidyl--serine, phosphatidyl--serine ) and phosphosphingolipids (sphingomyelin). In the former group the lipid tails are bound to a glycerol group while the latter do not have glycerol and the lipid tails have a backbone of sphingoid bases, absent in the former. These five types take into account for more then half of the lipids in most membranes.

The cell membrane has a quasi--liquid properties at physiological temperature. This is in part due to some disorder in the alignment of the lipid tails produced by the presence of unsaturated chains. Another contribution arise from the area occupied by the lipid heads which determines the distance between the hydrocarbon chains. This fluid character make the lipid bilayer like a solvent in which the other molecules are dissolved (lipids and proteins) and are free to diffuse. Moreover the lipids itself can move in different ways. The main movements and the associated time scales are summarized as follow
\begin{itemize}
	\item lipids conformational change (few nanoseconds);
	\item lipids protrusions out--of--plane (tens of picoseconds);
	\item diffusion within a leaflet (order of tens of nanoseconds);
	\item bilayer undulation and thickness involve collective motion of many lipids.
\end{itemize}
There are also many rare events that take place on the order of hours or even days, such like lipid flip--flop, in which a lipid go to the opposite leaflet; ions translocation; \textit{electroporation} by water, for example due to a cross membrane ions imbalance, in which water penetrate inside the bilayer and destroys a small portion in order to come to the other leaflet; water--helped ions permeation, called \textit{water--finger} and more general water defects inside the membrane.

For what concerns the length scales, the bilayer thickness is determined by the length of the lipid tails and their degree of unsaturation. Typically the hydrophobic region is $\sim 3$~nm thick while each hydrophilic regions is $\sim 1$~nm thick. Hence the typical bilayer thickness is around $\sim 4\div 5$~nm.

\subsection{Model cell membrane}
As we have seen above, the cell membrane is an extremely complex environment due to the large number of different biological molecules (lipids,proteins) which composes and resides in the membrane. The model membranes we will consider in this thesis will be composed of lipids only. This choice is dictated by three main reasons. First, current models and computational power could badly aim at reproducing the complexity of a real plasma membrane. Second, the use of a model system allows to take? fundamental questions concerning the physical and molecular  ?? interaction between \acp{NP} and membranes, last but not least, the model membrane we will consider resembles closely the model membranes used in a number of experimental and simulations results.  

In the bilayer model we will use, we consider model biological membrane consisting of \ac{POPC}; the chemical structure is shown in the top of figure~(\ref{fig:popc}). It is a zwitterionic glycerophospholipid of type phosphatidyl--choline whose head is made of a phosphate (PO$_4^-$) and a choline (C$_5$H$_{14}$NO$^+$) groups. It has two hydrocarbon chains: one is a saturated chain (palmitoyl) and the other is an unsaturated chain (oleoyl). The head groups and tails are both bounded to the glycerol group (C$_3$H$_8$O$_3$).
\begin{figure}[!ht]
	\centering
	\includegraphics[width=0.7\textwidth]{./img/POPC/popc}
	\caption{Top: chemical structure of a \acs{POPC} lipid. Bottom: \martini \acs{CG} model. The tan bead is the phosphate group, choline is in blue, the two pink beads represents the glycerol group and the hydrophobic chains in cyan.}
	\label{fig:popc}
\end{figure}

\paragraph{\textbf{CG model}} It is clear that the number of lipids that constitute a real cell membrane is 
enormous and it is impossible to take into account an entire cell membrane in a \ac{MD} simulation. A first 
approximation is to consider only a small area of the model bilayer. Given the medium area per lipid of about 
$0.65$~nm$^2$ and a portion of bilayer of $\sim 160$~nm$^2$, which corresponds to about $250$ lipids per leaflet, 
the total number of particles to be included in a atomistic simulation (excluding hydrogen atoms) are about 
$26\cdot 10^3$ plus the water molecules ($\sim 7\cdot 10^4$). This has a very expensive computational cost and 
the range of phenomena which can be studied on these time and length scales are very limited, calling for the 
adoption of a \ac{CG} approach.

\paragraph{\textbf{martini model}} As described in section \ref{sec:martini}, in this thesis work we will use the 
\ac{CG} \martini \ac{FF} for lipids \cite{Martini}.
%We consider a lipid bilayer made of $512$ \ac{POPC} lipids which extend on a surface of about $160$~nm$^2$ and whose thickness is about $4$~nm. 
The \martini model for the \ac{POPC} lipid maps the choline and the phosphate groups into two beads of type Q$_0$ 
and Qa negatively and positively charged, respectively. The saturated tail is modeled with four beads of type 
C$_1$ while the unsaturated tail is built up of four C$_1$ beads and one C$3$ bead which corresponds to the 
unsaturated group of atoms. The glycerol group is modeled with two beads of type N$_\text{a}$. A comparison 
between the chemical structure and \ac{CG} model is shown in figure~(\ref{fig:popc}).

\paragraph{\textbf{model accuracy}} The standard \martini \ac{FF} is able to capture the main physical properties 
of a lipid bilayer. These properties include the area per lipid, the distribution of groups across the membrane, 
the trend of the bending and the area compression moduli in function of the lipids composition and the 
unsaturation degree of the lipids, the stress profile across the membrane, the process of lipid desorption and 
flip--flopping, and many other as described in \cite{Martini} and \cite{MartiniReview}. Nevertheless many other 
properties, prevalently mediated by the electrostatic interaction, are not well described. As we have seen in the 
previous chapter this is because the \martini \ac{FF} does not take into account the long range treatment of the 
electrostatic interaction and because the standard \martini water is prevalently insensible to the electrostatic 
interaction\footnote{Moreover we can not forget that the \martini water bead takes into account four real water 
molecules. Hence the probability for a \martini water bead to permeate the hydrophobic region of the membrane is 
much less then for a water molecule in an atomistic \ac{FF}.}. To overcome to this problem the use of the \ac{PW} 
model and the \ac{PME} method, as outlined in \cite{MartiniReview} and \cite{PW}, are crucial to better describe 
the process that involve lipid bilayer, water and charged ions. These are ions translocation; electroporation of 
the membrane by water, due to a cross membrane ions imbalance; water--helped ions permeation and many other water 
defects inside the membrane as better described in the works of Marrink and Yesylevskyy.

\paragraph{\textbf{ions translocation}} An important phenomenon mediated by the electrostatic interactions that is crucial for this thesis work is the lipid membrane ions translocation. In \cite{PW} the authors have computed the \ac{FES} of the translocation of Na$^+$ and Cl$^-$ ions\footnote{The \martini model for Na$^+$ and Cl$^-$ associates the ion plus the hydration shell to a bead of type Qd positively charged and Qa negatively charged, respectively.} across a \acs{DPPC} membrane using umbrella sampling and the \ac{WHAM} with the standard \martini \ac{FF}, adding the \ac{PW} and adding together \ac{PW} and \ac{PME}. The height of the barriers are summarized in table~(\ref{tab:ionTranslocation}). The same \ac{FES} for a \acs{DMPC} membrane was calculated by Khavrutskii \etal\, \cite{atomisticTranslocation} with an atomistic \ac{FF}. Since the \martini model for the \acs{DPPC} lipid also model the \acs{DMPC} lipid a comparison can be made and it is shown in table~(\ref{tab:ionTranslocation}). As one can see, from left to right, increasing the loyalty of the treatment of the electrostatic interaction the \martini \ac{FF} approach the results of the atomistic \ac{FF}. Moreover in \cite{PW}, in accordance with the atomistic results in \cite{atomisticTranslocation}, for small cross membrane ions imbalance and only with the use of the \ac{PW} model and the \ac{PME} method, they observe some ion leakage without pore formation but still mediated by a water defect inside the membrane, called \textit{water finger} that help the ions to cross the hydrophobic region of the membrane. We remark that, these kind of phenomena, are totally absent using the standard \martini \ac{FF}. Hence, as already outlined, the importance to use a better treatment of the electrostatic interaction and a better model for water solvent.
\begin{table}[h!t]
	\centering
	\begin{tabular}{lcccc}
		\toprule
		\,		& Standard & \acs{PW} & \acs{PW} \& \acs{PME} & Atomistic	\\ \toprule
		Na$^+$	& 68.0	   & 67.6	  & 78.6					& 91.7 		\\ \midrule
		Cl$^-$	& 69.2	   & 70.4	  & 99.0					& 98.8		\\ \bottomrule
	\end{tabular}
	\caption{Height of the energy barrier (in kJ/mol) for Na$^+$ and Cl$^-$ translocation across a bilayer. The \martini results are based on a \acs{DPPC} membrane and are taken from \cite{PW} while the atomistic are based on a \acs{DMPC} membrane and are taken from \cite{atomisticTranslocation}.}
	\label{tab:ionTranslocation}
\end{table}

\section{NP--Membrane interaction}
Recently the literature for what concern the computational modeling about the interaction of such anionic monolayer--protected \ac{AuNP} and model lipid membranes has expanded contributing to sketch a possible mechanism of such interaction. The begging of the path way mechanism is the electrostatic attraction between the charged ligands and the polar head of the zwitterionic phospholipids in a fluid water phase: it is recognized to be the driving force for the adhesion of the \ac{AuNP} to the membrane surface, see figure~(\ref{fig:threeProcess}a). To the other end of the path way, it is known, on thermodynamic basis, that the most stable state for the \ac{AuNP} corresponds to the so--called \textit{snorkeling} configuration in which the \ac{NP} is embedded in the hydrophobic region of the membrane, while the charged ligands stably interact with the lipid heads of both leaflets, see figure~(\ref{fig:threeProcess}f). 

In the work of Federica Simonelli \etal\, \cite{ourPaper}, using the just described \ac{CG} \martini models of the \ac{NP} ligands and of biological membrane, with the standard \martini \ac{FF}, they found and characterize a possible path way mechanism of the interaction process with membrane at low curvature. Then they perform metadynamics simulations for characterize the thermodynamic of the process.

\subsection{Three--stage process}
%\paragraph{\textbf{Three--stage process}} 
The authors found a three--stage mechanism that regulate the insertion of the \ac{AuNP} into a low curvature membrane core: from the adsorbed state, figure~(\ref{fig:threeProcess}a), to the snorkeling configuration, figure~(\ref{fig:threeProcess}f). When the \ac{NP} in the water phase is approaching the surface of the membrane it enters in the Stage $1$ or in the adsorbed state in which the charged ligands interact with the polar lipid heads. Then, in Stage $2$, a key role is played by a lipid tail protrusion out of the hydrophobic region that stably interact with the hydrophobic beads of the \ac{NP} ligands, see figure~(\ref{fig:threeProcess}b-c). In order to minimize the contacts between water and the tails of the protruding lipid it pull the \ac{OT} ligands into the membrane core and the \ac{NP} enter in the hydrophobic contact state, figure~(\ref{fig:threeProcess}d). Then due the thermal fluctuation a charged bead is involved to cross the membrane core and stably interact with the head region of the opposite leaflet, figure~(\ref{fig:threeProcess}e).
\begin{figure}[ht!]
	\centering
	\subfloat[]{
		\includegraphics[width=4cm]{./img/threeProcess/a}
	}%
	\quad%
	\subfloat[]{
		\includegraphics[width=4cm]{./img/threeProcess/b}
	}%
	\quad%
	\subfloat[]{
		\includegraphics[width=4cm]{./img/threeProcess/c}
	}%
	\\%
	\subfloat[]{
		\includegraphics[width=4cm]{./img/threeProcess/d}
	}
	\subfloat[]{
		\includegraphics[width=4cm]{./img/threeProcess/e}
	}%
	\quad%
	\subfloat[]{
		\includegraphics[width=4cm]{./img/threeProcess/f}
	}%
	\caption{\acs{AuNP}--membrane interaction. (a) Stage $1$, adsorption of the \acs{NP} at membrane surface; (b) to (d) Stage $2$, the protrusion of a lipid tail initiates the hydrophobic contact that leads to partial embedding of the \acs{NP} in the membrane core; (e) the \ac{NP} binds to the opposite leaflet throwing a charged ligand; (f), Stage $3$, snorkeling configuration (five anchors shown). The hydrophobic beads of the ligands are shown in red and the charged beads in green. Lipid heads are blue (choline) and tan (phosphate), lipid tails are not shown, expect for (b) and (c), where the protruding lipid is shown with yellow tails. Water beads are not shown. All snapshot refer to a random (\ac{MUS}:\ac{OT} $1$:$1$) configuration. Taken from \cite{ourPaper}.}
	\label{fig:threeProcess}
\end{figure}
Since the anchored state is thermodynamically favorable respect to the hydrophobic contact, more and more ligands drop the charged beads to the head region of the opposite leaflet, approaching step--by--step the snorkeling configuration, figure~(\ref{fig:threeProcess}f). Moreover, compatibly with other works in literature, they found that the energy cost associated with the extraction of the \ac{NP} out of the membrane core is very high, making the anchoring process, and the snorkeling configuration, almost irreversible.
 
The authors performed \ac{MD} simulations for the three different \ac{NP} configurations showed in figure~(\ref{fig:coating}): the striped (\ac{MUS}:\ac{OT} $1$:$1$), the random (\ac{MUS}:\ac{OT} $1$:$1$) and the random (\ac{MUS}:\ac{OT} $2$:$1$). They found that the described behavior is common to the three configurations. Moreover, the lipid tail protrusion mechanism is validate form the kinematic of the simulations: the energy barrier to be overcome to move from stage $1$ to stage $2$ is of the order of the estimated energy cost of a lipid tail protrusion. Nevertheless, after the hydrophobic contact is reached, its lifetime depends on the surface ligand arrangements. For the random configurations the time lag between the hydrophobic contact and the first anchor is on the order of few nanoseconds. Instead, the striped configuration can linger in the stage $2$ for several microseconds. This suggest that the energy barrier to be overcome to move from stage $2$ to a one anchor state is less for the former than that for the latter. 

\subsection{Preliminary metadynamics results}
%\paragraph{\textbf{thermodynamics}} 
To quantify the thermodynamic behavior, metadynamics calculations are performed. In particular, in order to estimate the energy barrier for the anchor process, the \ac{FES} for only one anchored ligand, is computed for the three \ac{NP} configurations. The choice \ac{CV} is the $z$ component of the distance between the center of the charged beads and the \ac{COM} of \ac{POPC}. Then, starting from the hydrophobic state the metadynamics simulations begin biasing the charged bead. Since the achievement of the convergence it was not possible, the authors decide to perform different statistically independent metadynamics runs and stop each simulation at the recrossing process of the biased ligand. A recrossing is determined when the charged bead is $0.5$~nm above the \ac{COM} of the membrane in the leaflet in which the \ac{NP} stay. The resulting \ac{FES} profile, shown in figure~(\ref{fig:NPFES}), is obtained averaging the statistically independent runs. The errorbars are the standard error of the independent runs, as describe in \ref{sec:metadynamics}.
\begin{figure}[h!t]
	\centering
	\includegraphics[width=0.7\textwidth]{./img/NPFES}
	\caption{Free energy profile related to the transfer of a negatively charged ligand from the entrance leaflet to the opposite one. The blue is related to the striped (\ac{MUS}:\ac{OT} $1$:$1$) configuration while the red to the random (\ac{MUS}:\ac{OT} $1$:$1$) configuration. Metadynamics data are shown with errorbars, while the red dashed line are hypothesized profiles for the stage $2$ to stage $3$ transition of random \acp{NP}. Gray shades show the position of the polar head beads (AU). Taken from \cite{ourPaper}.}
	\label{fig:NPFES}
\end{figure}

For the striped arrangement, the crossing barrier to reach the saddle point, located at $0.5$~nm off the \ac{COM} of \ac{POPC}, is of the order of $\sim 22$~kJ/mol. While the recrossing energy barrier is approximately twice as high. The same is for the recrossing barrier of the random configuration. The absent of data for the random \acp{NP} in the hydrophobic contact is due to the difficulty to sample the stage $2$ to the anchor state transition for only one charged ligand. This is because, while the metadynamics is running on a specific charged ligand, other charged ligands spontaneously anchor to the opposite leaflet, making the free energy sampling of the first anchor impossible. Hence, together with the kinematic, for which the average lifetime of stage $2$ for random \acp{NP} is three order of magnitude shorter than for striped, the authors confirm the very low energy barrier, of the order of few $k_B T$, for the anchor process of the random \acp{NP}.


\subsection{Electrostatic interaction problem}
As we have outlined in the previous chapter, the standard \martini \ac{FF} poorly describe such process that strongly involve the electrostatic interaction. Moreover, the just describe anchoring process, in which one charged ligand drop its charged bead through the hydrophobic core of the membrane and anchors to the opposite leaflet, can be assimilated as an ion translocation process. A comparison of the energy barriers they obtained in \cite{ourPaper} and the data in table~(\ref{tab:ionTranslocation}) suggest that the standard \martini \ac{FF} can underestimate the energy barriers involved in the anchoring process. Thus, the necessity to increase the complexity of our model and try to use both \ac{PME} method and the \ac{PW} model. 
