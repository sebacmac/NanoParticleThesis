\documentclass[a4paper, 11pt]{article}
%\usepackage{pxfonts}
\usepackage{amsmath}
\usepackage{amssymb}
\usepackage[italian]{babel}
\usepackage[utf8]{inputenc}
\usepackage[T1]{fontenc}
\usepackage{graphicx}
\usepackage{hyperref}
\hypersetup{bookmarks}
\usepackage{gensymb}
\usepackage{mathrsfs}
\usepackage{cancel}
\usepackage{pgfplots}
\usepackage{tikz}
\usetikzlibrary{positioning}
\usetikzlibrary{decorations.pathmorphing}



\begin{document}

% \begin{tikzpicture}
% 	\draw[thin, gray!25] (0,0) grid (9,11);
% 	\node[rectangle,draw,rounded corners,very thin,minimum width=6cm,minimum height=1.5cm,text width=6cm,align=center] (A) at (0,10) {Initial conditions \\ and parameters are set};
% 	\node[rectangle,draw,rounded corners,very thin,minimum width=6cm,minimum height=1.5cm,text width=6cm,align=center] (B)  at (0,7.5) {Calculation of forces \\ from PEF is performed};
% 	\node[rectangle,draw,rounded corners,very thin,minimum width=6cm,minimum height=1.5cm,text width=6cm,align=center] (C)  at (0,5) {\footnotesize Equations of motion are integrated to obtain the new phase space vector};
% 	\node[rectangle,draw,rounded corners,very thin,minimum width=6cm,minimum height=1.5cm,text width=6cm,align=center] (D)  at (0,2.5) {Averages are computed \\ by equation~()};
% 	\node[rectangle,draw,rounded corners,very thin,minimum width=6cm,minimum height=1.5cm,text width=6cm,align=center] (E)  at (0,0) {END};
% 	\draw[thick,->] (A) -- (B);
% 	\draw[thick,->] (B) -- (C);
% 	\draw[thick,->] (C) -- (D);
% 	\draw[thick,->] (D) -- (E);
% 	\draw[-] (0,1.25) -- (5,1.25);
% 	\draw[-] (5,1.25) -- (5, 8.75) node[pos=0.5, fill=white] {$D$ Times};
% 	\draw[->] (5, 8.75) -- (0, 8.75);
% \end{tikzpicture}

% \begin{tikzpicture}
% 	\draw[step=1.5cm,thin] (-0.8,-0.8) grid (5.3,5.3);
% 	\draw[draw=red,ultra thick] (1.5,1.5) rectangle (3,3);
% 	\foreach \x in 	{0.6,2.1,3.6,5.1}
% 		\foreach \y in {0.3,1.8,3.3,4.8}
% 			\fill[fill=red] (\x,\y) circle (0.15);
% 	\foreach \x in 	{-0.4,1.1,2.6,4.1}
% 		\foreach \y in {-0.4,1.1,2.6,4.1}
% 			\fill[fill=blue] (\x,\y) circle (0.15);
% 	\foreach \x in 	{0.4,1.9,3.5,5}
% 		\foreach \y in {0.9,2.4,3.9,5.4}
% 			\fill[fill=green] (\x,\y) circle (0.15);
% \end{tikzpicture}

% \begin{tikzpicture}
% 	%\draw[thin, gray!25] (0,0) grid (15,8);
% 	\node[fill=red!70, circle, radius=0.15] (1) at (0,6) {};
% 	\node[fill=red!70, circle, radius=0.15] (2) at (2,8) {};
% 	\draw[decorate, decoration={coil,segment length=3mm,amplitude=2mm}] (1) -- (2);
% 	\node[] at (1.3,6.6) {$l_0$};
% 	\node[] at (1,5) {(a)};
% 	\node[fill=red!70, circle, radius=0.15] (3) at (4,8) {};
% 	\node[fill=red!70, circle, radius=0.15] (4) at (5.5,6) {};
% 	\node[fill=red!70, circle, radius=0.15] (5) at (7,8) {};
% 	\draw[thick,-] (3) -- (4) node[midway] (mid1) {};
% 	\draw[thick,-] (4) -- (5) node[midway] (mid2) {};
% 	\draw[-,decorate, decoration={coil,segment length=2mm,amplitude=2mm}] (4.8,6.91924) to[out=45,in=135] (6.2,6.9);
% 	\node[] at (5.5,7.8) {$\theta_0$};
% 	\node[] at (5.5,5) {(b)};
% 	\node[fill=red!70, circle, radius=0.15] (6) at (9, 8) {};
% 	\node[fill=red!70, circle, radius=0.15] (7) at (10, 6) {};
% 	\node[fill=red!70, circle, radius=0.15] (8) at (12, 6) {};
% 	\node[fill=red!70, circle, radius=0.15] (9) at (13, 8) {};
% 	\draw[thick,-] (6) -- (7);
% 	\draw[thick,-] (7) -- (8);
% 	\draw[thick,-] (8) -- (9);
% 	\draw[->] (11,5.9) arc [start angle=-160, end angle=160, x radius=0.16cm, y radius=0.28cm];
% 	\node[] at (11,5) {(c)};
% 	\node[fill=red!70, circle, radius=0.15] (10) at (3,4) {};
% 	\node[fill=red!70, circle, radius=0.15] (11) at (3.5, 2) {};
% 	\node[fill=red!70, circle, radius=0.15] (12) at (5.5, 3) {};
% 	\draw[thick,dashed] (10) -- (11);
% 	\draw[thick,dashed] (11) -- (12);
% 	\draw[thick,dashed] (12) -- (10);
% 	\node[] at (4.25,1) {(d)};
% 	\node[fill=red!70, circle, radius=0.15, label=above:$+q_2$] (13) at (8,4) {};
% 	\node[fill=red!70, circle, radius=0.15, label=left:$-q_1$] (14) at (8.5, 2) {};
% 	\node[fill=red!70, circle, radius=0.15, label=right:$-q_2$] (15) at (10.5, 3) {};
% 	\draw[thick,dashed] (13) -- (14);
% 	\draw[thick,dashed] (14) -- (15);
% 	\draw[thick,dashed] (15) -- (13);
% 	\node[] at (9.25,1) {(e)};
% \end{tikzpicture}

% \begin{tikzpicture}
% 	\begin{axis}[samples=1000,domain=0:3,restrict y to domain =-2:2,axis x line=bottom,axis y line=center,xlabel={$r$},xmin=0,ymin=-1.5,ymax=1.5,xmax=3,ytick={-1.5,-1,...,1.5}]
% 		\addplot[very thick]plot (\x,{4/\x^(12)-4/\x^(6)});
% 		\addplot[dashed]plot (\x, {0});
% 	\end{axis}
% 	\node[] at (-1.4,3) {$v(r)$};
% \end{tikzpicture}

% \begin{tikzpicture}
% 	\draw[gray,step=1.5] (0.1,0.1) grid (4.4,4.4);
% 	\fill[fill=red] (1.5,3) circle (0.25);
% 	\fill[fill=red] (3,1.5) circle (0.05);
% 	\fill[fill=red] (1.5,1.5) circle (0.15);
% 	\fill[fill=red] (3,3) circle (0.1);
%
% 	\fill[fill=black] (1, 3) circle (0.05);
% 	\fill[fill=black] (2, 2.8) circle (0.05);
% 	\fill[fill=black] (1.1, 3.5) circle (0.05);
% 	\fill[fill=black] (1.3, 2.4) circle (0.05);
% 	\fill[fill=black] (1.8, 3.6) circle (0.05);
%
% 	\fill[fill=black] (1.2, 1.9) circle (0.05);
% 	\fill[fill=black] (1.7, 1.3) circle (0.05);
% 	\fill[fill=black] (1.9, 1.9) circle (0.05);
%
% 	\fill[fill=black] (3.2,3.2) circle (0.05);
% 	\fill[fill=black] (2.8,3.4) circle (0.05);
%
% 	\fill[fill=black] (3.3,1.2) circle (0.05);
% \end{tikzpicture}

% \begin{tikzpicture}
% 	\begin{axis}[samples=1000,domain=-1:4,axis x line=center,axis y line=none,axis x line=none, xmin=-0.5,xmax=2.5,ymin=-3, ymax=3]
% 		\addplot[thick]plot (\x, {0});
% 		\addplot[]plot (\x, {exp(-9*\x*\x/0.1)});
% 		\addplot[]plot (\x, {exp(-9*(\x-2)^2/0.1)});
% 		\addplot[]plot (\x, {-exp(-9*(\x-1)^2/0.1)});
% 		\addplot[smooth] plot coordinates {(0,0) (0,-1.5)};
% 		\addplot[smooth] plot coordinates {(1,0) (1,1.5)};
% 		\addplot[smooth] plot coordinates {(2,0) (2,-1.5)};
% 	\end{axis}
% \end{tikzpicture}
%
% \begin{tikzpicture}
% 	\begin{axis}[samples=1000,domain=-1:4,axis x line=center,axis y line=none,axis x line=none, xmin=-0.5,xmax=2.5,ymin=-3, ymax=3]
% 		\addplot[thick]plot (\x, {0});
% 		\addplot[]plot (\x, {-exp(-9*\x*\x/0.1)});
% 		\addplot[]plot (\x, {-exp(-9*(\x-2)^2/0.1)});
% 		\addplot[]plot (\x, {exp(-9*(\x-1)^2/0.1)});
% 	\end{axis}
% \end{tikzpicture}



\end{document}
