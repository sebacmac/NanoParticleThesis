% !TEX root = ./../main.tex
\chapter{Results from metadynamics simulations}
%Introduzione al lavoro svolto durante la tesi per quanto rigurda le matadinamiche

%In my thesis work I primarily try to setting up \ac{MD} runs with the use of the previously described thiolated \ac{AuNP} model and the \ac{PME} method, and with \ac{PME} and the \ac{PW} model. By performing metadynamics runs involving the same anchoring process as in \cite{ourPaper} we can investigate if a more realist description can be obtained with a \ac{CG} model in order to use it for future applications. In order for a comparison to be made, at the same time, the PhD student Federica Simonelli, worked to set--up the same \ac{MD} and metadynamics runs with the use of an atomistic \ac{FF}, the best \ac{MD} description that we can obtain.

%Preliminary some test about the use of the \ac{PME} method and the \ac{PW} model with a simple \ac{POPC} bilayer was performed. Then metadynamics runs for obtaining the \ac{FES} of the anchoring process involving both the striped (\ac{MUS}:\ac{OT} $1$:$1$) and the random (\ac{MUS}:\ac{OT} $1$:$1$) \ac{NP} configurations, were performed with the use of the only \ac{PME} and the use of the \ac{PME} plus the \ac{PW} model. In view of the results, then, as better described in the next Chapter, unbiased \ac{MD} runs with the use of \ac{PME} and \ac{PW} involving all the \ac{NP} configurations were performed in order to investigate a possible change in the kinematic of the all the \ac{NP}--membrane interaction process.



\section{Metadynamics set--up}
%set up generale per una simulazione con metadinamica: procedura, start, stop e FES


%Initial configuration

%Metadynamics parameters

%Procedure

\section{Models compare}
%models compare (Standard, PME, PME & PW, Atomistico)

	%metadinamiche patched con PME e standard martini water

	%metadinamiche patched e random con PME e PW

\section{Convergence problems}

	%tentativi di convergenza con bound

\section{Head lipids dragging}
%Trascinamento delle teste dei lipidi

\section{Membrane leaflet engulfment}
%Risucchio delle membrane al recrossing